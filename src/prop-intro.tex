\chapter{Introduction}

Write a general introduction to static reflection.

What are the design goals and constraints
\begin{itemize}
\item Efficient compile-time evaluation.
\item Reflection values as template arguments.
\item Reasonably simple reflection API.
\item Absolutely zero impact memory footprint or runtime performance.
\end{itemize}

\section{Generative metaprograms}

Static reflection allows algorithms that operate on the semantic structure of entities within in a translation unit. 
A common use of these facilities is to generate a function definition based on the declarations within a class.
One such use is to generate a static reflection to generically define equality comparison: 

\begin{codeblock}
struct S { 
  int x, y, z;
}; 
 
int main() { 
 S s1 { 1, 0, 0 };
 S s2 { 0, 0, 1 };
 assert(equal(s1, s1));
 assert(!equal(s1, s2));
}
\end{codeblock}

The definition of \tcode{equal} can be defined by two reflective algorithms. The first, is the ``driver'', which sets up a call to an equality computation.

\begin{codeblock}
template<typename T> 
bool equal(const T& a, const T& b) {
  constexpr meta::object type = reflexpr(T);
  constexpr meta::object first = meta::first_member(type);
  return compare<first>(a, b);
}
\end{codeblock}

This calls a second function, \tcode{compare}, which will recursively compare the members of \tcode{a} and \tcode{b}. 
The \tcode{reflexpr} operator yields a token describing the (eventual) class substituted for \tcode{T}. 
The type of this value is \tcode{meta::info}. 
The \tcode{front} function yields a reflection of the member of the class, whatever that may be.

The \tcode{compare} function is somewhat more complex; it is shown below. 

\begin{codeblock}
template<meta::object X, typename T> 
bool compare(const T& a, const T& b) { 
  if constexpr (!meta::is_null(X)) { 
    if constexpr (meta::is_data_member(X)) { 
      auto p = valueof(X); 
      if (a.*p != b.*p) 
        return false; 
    }
    return compare<next(X)>(a, b); 
  } 
  return true; 
}
\end{codeblock}

Note that the algorithm is not \tcode{constexpr}; we require the function to have runtime behavior. 
Also, the compile-time values of the reflection are passed via template arguments. 
This is essential because the (syntactic) definition of this function depends on the meta-values passed into the function. 

We use \tcode{constexpr} if to selectively instantiate branches of the function: whenever the reflection X is non-null (i.e., does not correspond to an entity), we can continue recursing through the list. 
Otherwise, we have reached the end of the list of members, so the objects must be equal. 

If X is non-null and also a reflection of a data member, then we can compare the corresponding values of that data member in the objects a and b. 
The \tcode{valueof} operator yields a pointer-to-member value for the reflected data member. 
This can be used to extract and compare the actual values in \tcode{a} and \tcode{b}. 

In the implementation above, the reflection value must be passed as a template argument due to the use of the \tcode{valueof} operator in the definition.
This operator takes a constant expression operand (a reflection) and yields a value whose type is determined by the entity reflected.
Here, in each recursion through the function, \tcode{valueof(X)} will yield a pointer to a differently typed data member.  

An implementation using concepts, here presented using concepts, is more clearly specified, although a bit less compact. 

\begin{codeblock}
// Base case 
template<meta::info X, typename T> 
  requires meta::is_null(X)
bool compare(const T& a, const T& b) { 
  return true; 
} 
 
// Compares data members 
template<meta::info X, typename T> 
  requires meta::is_data_member(X)
bool compare(const T& a, const T& b) { 
  auto p = valueof(X); 
  if (a.*p != b.*p) 
    return false; 
  return compare<next(X)>(a, b); 
} 
 
// Skips non-data members 
template<meta::info X, typename T>
bool compare(const T& a, const T& b) { 
  return compare<next(X)>(a, b); 
} 
\end{codeblock}

Note that the algorithm cannot be written iteratively: 

\begin{codeblock}
for (auto m : meta::members(reflexpr(T))) { 
  if (meta::is_data_member(m)) 
    auto ptr = valueof(m); // error: m is not a constant expression 
}
\end{codeblock}

We further note that the earlier proposed expansion-based for loop\footnote{See P0589R90.} will also fail to satisfy the requirement because the loop variable cannot be declared \tcode{constexpr} (although that may be addressed in a future version). 

This algorithm is \emph{generative}. 
That is, the composition of the algorithm, not just its computation, depends on the values of reflected entities. 
In such cases, meta objects must be passed as template arguments. 
This distinction between normal and generative programming can make value-based reflection difficult to use; it requires implementers to understand when and how to context switch between the different processes of translation and evaluation \footnote{See P0992R0}.


\subsection{A non-generative metaprogram}

Not all reflective algorithms are generative (although we suspect that most useful algorithms are). Here is a simple example that counts (inaccurately) the number of entities nested within another. 

\begin{codeblock}
constexpr int count(meta::object x) { 
  int n = 0; 
  for (auto k : x) { 
    // See through templates... 
    if (meta::is_template(k)) 
      k = meta::templated_declaration(k); 

    // Count recursively 
    if (meta::has_members(k)) 
      n += count(k); 
  } 
  return 1 + n; 
}
\end{codeblock}

and it’s accompanying use: 

\begin{codeblock}
constexpr int n = count(reflexpr(std)); 
std::cout << n << '\n'; 
\end{codeblock}

The variable \tcode{n} must be \tcode{constexpr} in order to force compile-time evaluation of the count function. 

In order to eliminate the additional declaration, we introduce a new feature: immediate functions. An immediate function is one whose result is required to be computed at its call site. This is an extended form of constexpr, which requires a function to be evaluated at its call site: We could, for example, modify the definition of count like so: 

\begin{codeblock}
constexpr! int count(meta::object x) { 
  // Same as above 
}
\end{codeblock}

The \tcode{constexpr!} specifier marks the function as an immediate function. That lets us use the function more simply: 

\begin{codeblock}
std::cout << count(reflexpr(std)) << '\n'; 
\end{codeblock}

Note that, within the definition of count, the meta:: functions must also be immediate functions. We discuss issues around immediate functions in much more detail in Section XXX. 

\section{Interaction with other language features}

What features does reflection interact with?
\begin{enumerate}
\item What's the difference between concepts and reflections?
\item Can you reflect non-exported entities of a module? Failure to allow this will seriously limit the ability to generate algorithms outside of your own module.
\item Others?
\end{enumerate}
